\subsection{Installing haskell}

\begin{frame}[fragile]{Using binaries}

  \begin{itemize}
  \item For Windows: Download and extract
    this
    \href{https://downloads.haskell.org/~ghc/9.2.5/ghc-9.2.5-x86_64-unknown-mingw32-integer-simple.tar.xz}{tarball}.
  \item For MacOs: Available binaries are
    in \url{https://www.haskell.org/ghc/download_ghc_9_4_4.html}. Not
    recommended as libraries are 
    not automatically linked and default Apple privacy settings will make hard
    their interaction. 
  \item For Linux: Download and extract
    this
    \href{https://downloads.haskell.org/~ghc/9.4.4/ghc-9.4.4-x86_64-fedora33-linux.tar.xz}{tarball}. Inside
    the extracted directory execute the \verb|configure| script and then
    \verb|sudo make install|.
  \end{itemize}

Note: This approach is not recommended since the installation scripts do not
provide an easy way to uninstall the installed files.

\end{frame}

\begin{frame}[fragile]{Using a package manager}
  \begin{itemize}
  \item For MacOs:
    \begin{itemize}
    \item Using Homebrew: \verb|brew install ghc|
    \item Using MacPorts: \verb|sudo port install ghc|
    \end{itemize}
  \item For Linux:
    \begin{itemize}
    \item Debian/Ubuntu: \verb|sudo apt-get install ghc|
    \item Arch Linux: \verb|sudo pacman -S ghc|
    \item Fedora: \verb|sudo dnf install ghc|
    \item openSUSE: \verb|sudo zypper install ghc|
    \end{itemize}
  \end{itemize}
\end{frame}